\documentclass[onecolumn, draftclsnofoot,10pt, compsoc]{IEEEtran}
\usepackage{graphicx}
\usepackage{url}
\usepackage{setspace}

\usepackage{geometry}
\geometry{textheight=9.5in, textwidth=7in}

% 1. Fill in these details
\def \CapstoneTeamName{		Ancestry Data Viewer}
\def \GroupMemberOne{			Le-Chuan Justin Chang}
\def \GroupMemberTwo{			Monica Sek}
\def \GroupMemberThree{			Yong Ping Li}
\def \CapstoneProjectName{		Ancestry Data Viewer}
\def \CapstoneSponsorCompany{	ADVR}
\def \CapstoneSponsorPerson{		Ashley McGrath}

% 2. Uncomment the appropriate line below so that the document type works
\def \DocType{		%Problem Statement
				%Requirements Document
				Technology Review
				%Design Document
				%Progress Report
				}
			
\newcommand{\NameSigPair}[1]{\par
\makebox[2.75in][r]{#1} \hfil 	\makebox[3.25in]{\makebox[2.25in]{\hrulefill} \hfill		\makebox[.75in]{\hrulefill}}
\par\vspace{-12pt} \textit{\tiny\noindent
\makebox[2.75in]{} \hfil		\makebox[3.25in]{\makebox[2.25in][r]{Signature} \hfill	\makebox[.75in][r]{Date}}}}
% 3. If the document is not to be signed, uncomment the RENEWcommand below
%\renewcommand{\NameSigPair}[1]{#1}

%%%%%%%%%%%%%%%%%%%%%%%%%%%%%%%%%%%%%%%
\begin{document}
\begin{titlepage}
    \pagenumbering{gobble}
    \begin{singlespace}
    	\includegraphics[height=4cm]{coe_v_spot1}
        \hfill 
        % 4. If you have a logo, use this includegraphics command to put it on the coversheet.
        %\includegraphics[height=4cm]{CompanyLogo}   
        \par\vspace{.2in}
        \centering
        \scshape{
            \huge CS Capstone \DocType \par
            {\large\today}\par
            \vspace{.5in}
            \textbf{\Huge\CapstoneProjectName}\par
            \vfill
            {\large Prepared for}\par
            \Huge \CapstoneSponsorCompany\par
            \vspace{5pt}
            {\Large\NameSigPair{\CapstoneSponsorPerson}\par}
            {\large Prepared by }\par
            Group 22\par
            % 5. comment out the line below this one if you do not wish to name your team
            \CapstoneTeamName\par 
            \vspace{5pt}
            {\Large
                \NameSigPair{\GroupMemberOne}\par
                \NameSigPair{\GroupMemberTwo}\par
                \NameSigPair{\GroupMemberThree}\par
            }
            \vspace{20pt}
        }
        \begin{abstract}
        % 6. Fill in your abstract    
        This document is describes the technologies needed to be used to create the viewer.
        \end{abstract}     
    \end{singlespace}
\end{titlepage}
\newpage
\pagenumbering{arabic}
%\tableofcontents
% 7. uncomment this (if applicable). Consider adding a page break.
%\listoffigures
%\listoftables
%\clearpage

% 8. now you write!
\section{1.1 Overview}
The User Interface (UI) is what the user will interface with in order to further examine and access additional features in the Ancestry Data Viewer. The UI should, as dictated by the requirements document, be both easy to use, and appear at least somewhat aesthetically pleasing.

\section{1.2 Criteria}
The criteria for the tool used is that is must be compatible with both Linux, specifically Ubuntu, and Windows 10. This is the most important criteria for the tool. Another important criteria is how well the tool can be used to create UI, since one of the criteria as referenced in the requirements document is that the UI must be aesthetically pleasing and easy to use. Another criteria is that the UI choice must be compatible with the tool or tools being used for displaying the family tree.

Criteria that is used in the event of equally suited tools will simply be to select whichever tool is easiest to use. 

It should be noted that the primary component to fulfilling the usability and aesthetics component of the UI is more likely to be the result of design rather then the tool used to develop it, however, the selected tool is an important component for ease of development.

\section{1.3 Potential Choices}
\section{1.3.1 JUCE}
JUCE is a partially open-source library that uses OpenGL to generate a user interface. Since it runs on OpenGL, it is compatible with both Windows and Linux, and as an added bonus it is also compatible with Mac [1]. Another feature of JUCE is that it is also compatible with mobile devices [1]. JUCE is free for personal use, though a license is required for larger teams and larger applications. One notable source of JUCE’s users are creators of music creation tools, such as FL Studios [2]. There are also various tutorials for using JUCE available on its website.

A helpful inclusion with JUCE is the Projucer, which is an application that helps with developing UI. By allowing the developer to manipulate the GUI by hand, be it resizing or changing the colors [3]. It can also allow HTML-like development, except that it allows C++. Projucer is also compatible with various IDEs, such as Visual Studios [3].

Another aspect of JUCE is that it supports plug-in development, especially for VST, AU and AAX [1]. It is frequently used for various plug-ins for various Digital Audio Workstation. Additionally, JUCE is compatible with mobile devices, both Android and iOS [4]. An aspect that is less direct but helpful and notable nonetheless is that there is a forum for developers, in case assistance is needed from experienced developers.

\section{1.3.2 UMG UI Designer}
UMG, short for Unreal Motion Graphics User Interface Designer, is a UI development tool for Unreal Engine [5]. It is a visual editor that allows for the creation of menus, and other interface graphics [5]. UMG is bundled and designed for use with Unreal Engine, so there is little need for an external UI library if Unreal is selected to be the exclusive visual development format. 

UMG has an extensive tutorial available online, and it features an online forum to request assistance.

\section{1.3.3 Unity UI}
Unity includes a UI creator, which features a tutorial and forums to consult if needed [6]. The Unity UI creator is frequently used for various Unity games. There is a problem in that Unity UI is very resource intensive, to the extent that it runs twice as often as the default Update() function from Unity. The Unity UI creator is also a visual editor.

\section{1.4 Discussion}
For the purposes of developing the Ancestry Data Viewer, there is no glaring flaws with any of the three technologies. JUCE works with every application, while Unity and Unreal are optimized for their respective engines. 

\section{1.5 Conclusion}
We have concluded that the tool used to create the UI is heavily dependent upon what application or applications are used to generate the image. For instance, if we are fully committed to Unreal Engine for every part of visualization, then it would be simple to simply use UMG, on the grounds that there would be no concern for compatibility, since it is built in. Alternatively, there is not a particular need to use it if Unreal Engine is not in use. Therefore, the UI framework would be Unity UI if the Unity Game Engine is used, UMG if it's Unreal, and JUCE if it's neither.

\section{4 References}
[1]https://juce.com/features \newline
[2]https://juce.com/made-with-juce \newline
[3]https://juce.com/projucer \newline
[4]https://juce.com/maq \newline
[5]https://docs.unrealengine.com/latest/INT/Engine/UMG/ \newline
[6]https://unity3d.com/learn/tutorials/s/user-interface-ui \newline
[7]https://forum.unity.com/threads/ui-design-tool-pros-vs-cons-unitygui-guitextures-3d-mesh-or-3rd-party-tools.76937/

\end{document}