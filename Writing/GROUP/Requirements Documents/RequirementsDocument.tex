\documentclass[onecolumn, draftclsnofoot, 10pt, compsoc]{IEEEtran}
\usepackage{graphicx}
\usepackage{url}
\usepackage{setspace}

\usepackage{geometry}
\geometry{textheight=9.5in, textwidth=7in}

% 1. Fill in these details
\def \CapstoneTeamName{		Team Ancestry Data Viewer(ADVR)}
\def \CapstoneTeamNumber{		22}
\def \GroupMemberOne{			YongPing Li}
\def \GroupMemberTwo{			Monica Sek}
\def \GroupMemberThree{			Le-Chuan Chang}
\def \CapstoneProjectName{		Ancestry Data Viewer}
\def \CapstoneSponsorCompany{	}
\def \CapstoneSponsorPerson{		Ashley McGrath}

% 2. Uncomment the appropriate line below so that the document type works
\def \DocType{	%Problem Statement
				Requirements Document
				%Technology Review
				%Design Document
				%Progress Report
				}
			
\newcommand{\NameSigPair}[1]{\par
\makebox[2.75in][r]{#1} \hfil 	\makebox[3.25in]{\makebox[2.25in]{\hrulefill} \hfill		\makebox[.75in]{\hrulefill}}
\par\vspace{-12pt} \textit{\tiny\noindent
\makebox[2.75in]{} \hfil		\makebox[3.25in]{\makebox[2.25in][r]{Signature} \hfill	\makebox[.75in][r]{Date}}}}
% 3. If the document is not to be signed, uncomment the RENEWcommand below
%\renewcommand{\NameSigPair}[1]{#1}

%%%%%%%%%%%%%%%%%%%%%%%%%%%%%%%%%%%%%%%
\begin{document}
\begin{titlepage}
    \pagenumbering{gobble}
    \begin{singlespace}
    	%\includegraphics[height=4cm]{coe_v_spot1}
        \hfill 
        % 4. If you have a logo, use this includegraphics command to put it on the coversheet.
        %\includegraphics[height=4cm]{CompanyLogo}   
        \par\vspace{.2in}
        \centering
        \scshape{
            \huge CS Capstone \DocType \par
            {\large\today}\par
            \vspace{.5in}
            \textbf{\Huge\CapstoneProjectName}\par
            \vfill
            {\large Prepared for}\par
            \Huge \CapstoneSponsorCompany\par
            \vspace{5pt}
            {\Large\NameSigPair{\CapstoneSponsorPerson}\par}
            {\large Prepared by }\par
            Group\CapstoneTeamNumber\par
            % 5. comment out the line below this one if you do not wish to name your team
            \CapstoneTeamName\par 
            \vspace{5pt}
            {\Large
                \NameSigPair{\GroupMemberOne}\par
                \NameSigPair{\GroupMemberTwo}\par
                \NameSigPair{\GroupMemberThree}\par
            }
            \vspace{20pt}
        }
        %\begin{abstract}
        % 6. Fill in your abstract    
        	 
        %\end{abstract}     
    \end{singlespace}
\end{titlepage}
\newpage
\pagenumbering{arabic}
\tableofcontents
% 7. uncomment this (if applicable). Consider adding a page break.
%\listoffigures
%\listoftables
\clearpage

% 8. now you write!
\section{Introduction}
\subsection{Purpose}
\begin{singlespace}
The purpose of this project is to create a program that will open GEDCOM files and show the data contained in a clear and easy to read manner.
\end{singlespace}

\subsection{Scope}
\begin{singlespace}
The software will open and get data from GEDCOM file and use the data to produce the required features. These features include full tree view, direct lineage view, 3D view, and also the ability to find the common ancestor of two people(node) in the tree. 
\end{singlespace}

\subsection{Definitions, acronyms, and abbreviations}
\begin{singlespace}
Tree : is actually refers to the tree data structure for displaying data. Elements in the tree is refer to as nodes. Every person’s name in the GEDCOM file will be a node on our tree diagram.
\newline
\newline
Graph: A graph is a superset of Trees. Whereas a node trees can hold only two children per node, a graph is capable of holding as many nodes as desired.
\end{singlespace}

\subsection{References}
\begin{singlespace}

\end{singlespace}

\subsection{Overview}
\begin{singlespace}
 Our project will create an application that is capable of reading and parsing GEDCOM files. After parsing the files, it will generate a graph, with each node representing one member of the family. The graph will be arranged in a legible fashion, which displays each person chronologically and connects them. Our project will also allow users to quickly search for common ancestors, and see the direct lineage of a given person. Finally, our project will allow the user to view it in VR if they have a headset available.
\end{singlespace}

\section{Overall Description}
\subsection{Product perspective}
\begin{singlespace}
This project is to create an Ancestry data viewer for our client to use. It’s also available for everybody out there because the code is open sources(GITHUB). The user will interface with our software through a laptop or desktop with Linux or Windows operating system. The user will need to have their own GEDCOM file and open the GEDCOM file in our software. The software will run an algorithm to obtain the data and display it in a clear and easy to read manner.
\end{singlespace}

\subsection{Product functions}
\begin{singlespace}
Our product should be capable of reading GEDCOM files, then displaying them. It should also be capable of displaying every person included in the GEDCOM file, and should be able to filter out people who aren’t directly related to the selected person. The application will highlight the nearest ancestor of two selected people.
\end{singlespace}

\subsection{User characteristics}
\begin{singlespace}
Our user is Ashley McGrath, who is our client. While our users is likely to be standard denizens of the Internet, we are primarily designing our application for our client. 
\newline
\newline
A standard user should not need to have great technical knowledge in order to successfully use the application. They should be able to read and understand basic computer terminology, such as clicking, and moving the mouse, and should be able to read and follow simple instructions.

\end{singlespace}

\subsection{Constraints}
\begin{singlespace}
We are required to create a 3D view of the family tree. The 3D view can only be accessed with the VR devices. 
\end{singlespace}

\subsection{Assumptions and dependencies}
\begin{singlespace}
We are dependent on pre-existing VR frameworks in order to implement VR functionality. If not, we will not be able to create our own, as this would be too complicated and time-consuming.
\end{singlespace}

\section{Specific requirements}
\begin{itemize}
\item Our application must be capable of opening and displaying GEDCOM files.
\item Our application must be able to run on Linux and Windows operating systems. 
\item Our application must be VR compatible
\item Our application must include the ability to find the nearest ancestor of two given people.
\item Our application must be able to highlight direct relatives of a given person
\end{itemize}

\appendix
%\section{Appendixes}
%\index{a}
%\section{Index}
\end{document}