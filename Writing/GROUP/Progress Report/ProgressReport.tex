\documentclass[onecolumn, draftclsnofoot, 10pt, compsoc]{IEEEtran}
\usepackage{graphicx}
\usepackage{url}
\usepackage{setspace}
\usepackage{hyperref}

\usepackage{geometry}
\geometry{textheight=9.5in, textwidth=7in}

% 1. Fill in these details
\def \CapstoneTeamName{		Team Ancestry Data Viewer(ADVR)}
\def \CapstoneTeamNumber{		22}
\def \GroupMemberOne{			YongPing Li}
\def \GroupMemberTwo{			Monica Sek}
\def \GroupMemberThree{			Le-Chuan Chang}
\def \CapstoneProjectName{		Ancestry Data Viewer}
\def \CapstoneSponsorCompany{	}
\def \CapstoneSponsorPerson{		Ashley McGrath}

% 2. Uncomment the appropriate line below so that the document type works
\def \DocType{	%Problem Statement
				%Requirements Document
				%Technology Review
				%Design Document
				Progress Report
				}
			
\newcommand{\NameSigPair}[1]{\par
\makebox[2.75in][r]{#1} \hfil 	\makebox[3.25in]{\makebox[2.25in]{\hrulefill} \hfill		\makebox[.75in]{\hrulefill}}
\par\vspace{-12pt} \textit{\tiny\noindent
\makebox[2.75in]{} \hfil		\makebox[3.25in]{\makebox[2.25in][r]{Signature} \hfill	\makebox[.75in][r]{Date}}}}
% 3. If the document is not to be signed, uncomment the RENEWcommand below
%\renewcommand{\NameSigPair}[1]{#1}

%%%%%%%%%%%%%%%%%%%%%%%%%%%%%%%%%%%%%%%
\begin{document}
\begin{titlepage}
    \pagenumbering{gobble}
    \begin{singlespace}
    	%\includegraphics[height=4cm]{coe_v_spot1}
        \hfill 
        % 4. If you have a logo, use this includegraphics command to put it on the coversheet.
        %\includegraphics[height=4cm]{CompanyLogo}   
        \par\vspace{.2in}
        \centering
        \scshape{
            \huge CS Capstone \DocType \par
            {\large\today}\par
            \vspace{.5in}
            \textbf{\Huge\CapstoneProjectName}\par
            \vfill
            {\large Prepared for}\par
            \Huge \CapstoneSponsorCompany\par
            \vspace{5pt}
            {\Large\NameSigPair{\CapstoneSponsorPerson}\par}
            {\large Prepared by }\par
            Group\CapstoneTeamNumber\par
            % 5. comment out the line below this one if you do not wish to name your team
            \CapstoneTeamName\par 
            \vspace{5pt}
            {\Large
                \NameSigPair{\GroupMemberOne}\par
                \NameSigPair{\GroupMemberTwo}\par
                \NameSigPair{\GroupMemberThree}\par
            }
            \vspace{20pt}
        }
        \begin{abstract}
        % 6. Fill in your abstract    
		The purpose of our project is to read in a GEDCOM file and display its information onto an ancestry tree. The purpose of this progress report is to illustrate what we have done in the past 10 weeks. Identify any problems we have impeded and did we found the solutions to the problems.
        \end{abstract}     
    \end{singlespace}
\end{titlepage}
\newpage
\pagenumbering{arabic}
\tableofcontents
% 7. uncomment this (if applicable). Consider adding a page break.
%\listoffigures
%\listoftables
\clearpage

% 8. now you write!
\section{Project Overview }
\begin{singlespace}
The purpose of our project is to read in a GEDCOM file and display its information onto an ancestry tree. The GEDCOM file needs to be parsed through to make it easier for the software to read and to gather the required data. The ancestry tree will be displayed in a clear graph that contains series of parent and children nodes. The software has two applications: desktop and VR. A user should be able to toggle between desktop and VR mode if a VR headset is presented. In addition, the following special features should be included in the software: choosing a subsection of the tree, finding a common ancestor between two nodes (members), and toggling between a full ancestry tree and a direct lineage tree.
\end{singlespace}

\section{Current Status}
\begin{singlespace}
Before we talk about our current status, we should briefly talk about what we have completed so far. We mainly wrote four documents for this project: problem statement, requirements document, technology review and design document.
\begin{itemize}
\item The problem statement is to identify the purpose of the project and get an idea of why our client had us build this software.
\item The requirements document is to identify functionalities required by the client for the software to develop and slightly touch on what we need to do for the project. This document basically displays what the client wants as the result of our project. 
\item The technology review is to identify the different parts of the software and research on tools we can use for the different parts.
\item The design document is to illustrate how the different parts of the technology review are put together and create our software. We have to justify our choices for each part to support our design. (The client could access these documents because we required signatures for the documents.)
\end{itemize}
 

After we finish the documents, we have designed what we envision our software to look like and have a basic understanding of how we plan on doing it. We are currently transitioning from documentation to development. We could start on the development after our client approves our design documents. Our next course of action would be implementing our parser code and to get familiar with display features on Unreal Engine. Since it is not productive for all of us be working on the parser and we don't have a working parser ready, we will be using some fake data to test our implementation of visualization algorithm, etc.
\end{singlespace}

\section{Problems}
\begin{singlespace}
We did not have a particularly large amount of major issues over the term, but we had some stumbling blocks that should be noted.
One of our stumbling blocks was that we weren't meeting word counts. This was mainly due to the fact that we were not starting the assignment early enough. We prioritized rough drafts less because they were rough drafts, so it wouldn't hurt us very much if we turned in poor drafts. This worked adequately for this term, which is based around writing, and was still reasonably structured, but will likely be a problem in later terms, where coding blocks will inhibit progress.

Another problem was that our TeX files were not being created correctly from the makefiles on the server. When we were creating it, it would behave as though the margins were too large and cut parts of our final PDFs out. We were eventually informed that this was due to a problem with how our makefile was configured. Specifically, we were trying to use a GUI latex editor, but our given TeX file was designed for a different compiler.

An issue that arose after we met with our client. We realized that there were several edge-cases that, while currently harmless, will require adjustment later. For instance, certain families have adoption, divorce, or incest. We were informed to treat divorce as married but with a different line, try to make incest look reasonable on the tree, and that our client would decide what to do about adoption later.

Finally, our largest problem was related to the technology review assignment. We had difficulty identifying nine different pieces of technology. In order to solve this problem, we went in to office hours, and to our TA during a meeting, and managed to figure out most of our technologies. After some brainstorming, we figured out what our final technology should be.
\end{singlespace}

\section{Weekly Reports}
\subsection{Week 1}
\subsubsection{Plans}
During our first week, we were focused on completing our preference assignment.
\subsubsection{Problems}
No problems this week.
\subsubsection{Progress}
We finished our preference assignments.


\subsection{Week 2}
\subsubsection{Plans}
During the second week, we planned on contacting our client, scheduling a meeting, and meeting the rest of our team.
\subsubsection{Problems}
No problems this week.
\subsubsection{Progress}
We met with each other, then sent our client an email, and proposed a meeting time.

\subsection{Week 3}
\subsubsection{Plans}
For the third week, we needed to meet with our client and work on a rough draft of our problem statement.
\subsubsection{Problems}
None of our drafts met the 1000 word requirement.
\subsubsection{Progress}
We all wrote a rough draft, and we all met with our client over Google Hangouts.

\subsection{Week 4}
\subsubsection{Plans}
On the fourth week, we were supposed to have our first meeting with our TA, and we were supposed to finish the problem statement.
\subsubsection{Problems}
No particular problems this week, though due to a miscommunication, only one member of the group met with the TA.
\subsubsection{Progress}
One member of the team met with the TA and conveyed what the meeting was for to the rest of the group, and the rough draft was revised and submitted. 

\subsection{Week 5}
\subsubsection{Plans}
For the fifth week, we needed to work on the requirements document.
\subsubsection{Problems}
There were no particularly large problems for the fifth week, though we missed the word count on the rough draft.
\subsubsection{Progress}
We successfully completed the rough draft for the requirements document.

\subsection{Week 6}
\subsubsection{Plans}
We had another meeting with our client, and we needed to finish the requirements document.
\subsubsection{Problems}
For the sixth week, there were no major problems to report.
\subsubsection{Progress}
We clarified various questions about the final result of the project and some requirements with our client, and we revised and submitted the requirements document.

\subsection{Week 7}
\subsubsection{Plans}
We need to start on technology review.

\subsubsection{Problems}
Coming up with nine components to the software.

\subsubsection{Progress}
During week 7, we started on our technology review document after we finished the requirements document. We met as a group and come up with different components for the software, but we weren't able to come up with nine components. We went to our instructor's office hour and got a few helpful advice from him, but still didn't meet the required nine components. We went on our own after we distributed the components because the technology review is an individual assignment. Each of us researched on tools for each component that we are in charge of but we also mention that we have the option of implementing our own tool for some components.  


\subsection{Week 8}
\subsubsection{Plans}
Do more research on the components each of us are in charge of and continue working on the technology review document.

\subsubsection{Problems}
Coming up with nine components to the software.
\subsubsection{Progress}
During week 8, we finished our research on the components that we came up with. We chose a tool for each component and justified our choice base on the research we have done. We continue to brainstorm ideas for the components of the software but didn't come up with anything. On Tuesday of this week, we paired review with our other classmates and revise the document. We met with our TA on Wednesday and he helped us come up with more components to the software and we got our nine required components. We then did more research on the new components separately.

\subsection{Week 9}
\subsubsection{Plans}
Do more research on the components each of us are in charge of and finish the technology review document.
\subsubsection{Problems}
There were no problems for the week 9.
\subsubsection{Progress}
At the beginning of week 9, we are mostly finished with research and our technology review document. We finalized and revised our technology review document. The technology review document is turned in on Tuesday. We sent a copy to our instructor and uploaded to GitHub. We didn't have class this week due to Thanksgiving. We didn't do more work for this class and enjoyed our long holiday weekend. 

\subsection{Week 10}
\subsubsection{Plans}
We are meet and write the design document.
\subsubsection{Problems}
There were no problems on week 10.
\subsubsection{Progress}
During week 10, we met as a group and went to class to look at examples of the design document. Afterwards, we discussed how to put the components together and develop our design document. After the meeting, we each wrote about the different components that we are in charge of. For each component, we have to write about its functionalities, how it is implemented, and justified why we chose to implement it this way. We added our introduction and conclusion after we finish our parts and turned in the design document. The design document is uploaded to GitHub.

\section {Retrospective}
\begin {tabular} { | p{0.3\linewidth} | p{0.3\linewidth} | p{0.3\linewidth} | }
\hline
Positives & Deltas & Actions \\
\hline
a: &a &a \\

\hline
\end {tabular}




\end{document}
